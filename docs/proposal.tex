\documentclass[10pt,a4paper]{report}
\usepackage[utf8]{inputenc}
\usepackage{amsmath}
\usepackage{amsfonts}
\usepackage{amssymb}
\author{Benjamin}
\title{Project Proposal}
\begin{document}
\section{Introduction}
For a long time, pilots have struggled with maps. The problem comes with the territory - maps are big, and cockpits are small. The obvious solution lies in applications like Google Maps, where one can smoothly browse through the maps at will. Application developers have responded to this requirement, making things called \emph{electronic flight bags}, which can contain all of the charts and maps that a pilot needs. 

However, these come at a substantial price. A commercial EFB might cost in upwards of two thousand dollars, and EFBs using existing hardware (such as iPads and Android tablets) are little better once the cost of the tablet is factored in. We want to fix this by providing a free alternative application for Android that has all of the functionality but none of the costs.

Fundamentally, this application will allow pilots to access their maps more quickly and easily, especially when compared to paper charts. On top of this foundation, we will also provide what is known as \emph{flight planning} functionality, where a pilot enters his desired route into the device and it plots it on the map. The information for this view is given by industry standard data sources and methods, allowing safe-for-flight information to be displayed.

Our working name for this application is InFlight, encapsulating the use of the application for pilots and the kind of information it provides.
\section{Plan}
TODO
\section{Benefits}
Pilots are constantly fighting with their maps. Imagine yourself in a light aircraft cockpit - it's about 50\% narrower than a car - with another person just millimeters from your right elbow. In addition, the yoke, the way the airplane is controlled, is just inches from your chest, and you can't touch it without altering the airplanes direction. Amongst all of this, think about an aviation map. The standard "sectional" (regional aviation map) is nearly 3 feet square, and needs to be entirely unfolded and examined to find the location you're looking for. Throw in radio communication, and one has a recipe for at least confusion, and at most disaster. 

In recent years, devices known as "Electronic Flight Bags" have become popular. The first devices were custom-made tablets with FAA certifications that provided everything and the metaphorical kitchen sink. This functionality came at very large cost, however, with prices reaching into the low thousands for the device alone, never mind the mandatory support.

Then, into this world of overpriced EFBs came the iPad and Android tablets. These provided the hardware required to run EFB applications, and run they did. Many flight-management applications popped up on the various devices, such as Naviator and ForeFlight, but there was (and is) the same constant: price. All of these applications cost quite a lot to keep running - ForeFlight, for instance, costs 120\$ a year. While more reasonable than the dedicated hardware that it replaced, it's still quite expensive, and we think we can fix this.

We plan to provide an inexpensive alternative application (~10-20\$/year) that has near feature parity and similar performance to the "big name" applications. This, in conjunction with deployment onto Android devices, generally cheaper than the iPad, will allow more pilots to use the application.

\section{Approach}
We intend to use the Android SDK as our primary development environment. This is (for obvious reasons) the industry standard mechanism for creating and maintaining Android applications. By necessity, then, we will be developing in Java 7 running on this platform. 

The core of our application is the Federal Aviation Administration's publicly available datasets. These include comprehensive charts of all kinds, as well as safe-for-navigation data that allows us to provide up-to-date data quickly. As a safety critical application, this is essential to use of our software.

\section{Evaluation}
We intend to use a combination of use testing by several active pilots, as well as ground usability testing to evaluate application performance. 

\section{Qualifications}
\begin{itemize}
\item \emph{Benjamin Chung} is a student pilot with more than 40 hours, as well as an academic software engineer specializing in compiler implementation. He has more than 5 years of experience with the Java platform and 3 with the Android platform.
\end{itemize}

\end{document}